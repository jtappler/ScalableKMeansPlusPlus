\documentclass[fontsize=11pt]{scrartcl}
\usepackage[T1]{fontenc}
\usepackage{fourier}
\usepackage[english]{babel}
\usepackage{amsmath,amsfonts,amsthm}
\usepackage{lipsum}
\usepackage{sectsty}
\allsectionsfont{\centering \normalfont\scshape} % Make all sections centered, the default font and small caps

\usepackage{fancyhdr} % Custom headers and footers
\pagestyle{fancyplain} % Makes all pages in the document conform to the custom headers and footers
\fancyhead{} % No page header - if you want one, create it in the same way as the footers below
\fancyfoot[L]{} % Empty left footer
\fancyfoot[C]{} % Empty center footer
\fancyfoot[R]{\thepage} % Page numbering for right footer
\renewcommand{\headrulewidth}{0pt} % Remove header underlines
\renewcommand{\footrulewidth}{0pt} % Remove footer underlines
\setlength{\headheight}{13.6pt} % Customize the height of the header
\numberwithin{equation}{section}
\numberwithin{figure}{section}
\numberwithin{table}{section}
\setlength\parindent{0pt}

%----------------------------------------------------------------------------------------
%	TITLE SECTION
%----------------------------------------------------------------------------------------

\newcommand{\horrule}[1]{\rule{\linewidth}{#1}} % Create horizontal rule command with 1 argument of height

\title{	
\normalfont \normalsize 
\horrule{0.5pt} \\[0.4cm]
\huge Algorithm (Pseudocode) \\
\horrule{2pt} \\[0.1cm]
}
\begin{document}
\date{}
\maketitle % Print the title


\section*{Notations}

Let $X =\{x_1,...x_n\}$ be the set of points in $d$-dimensional Euclidean space, and let $k$ be a positive integer specifying the number of clusters. Let ||$x_i$ - $x_j$ || denote the Euclidean distance between $x_i$ and $x_j$. For a point $x$ and a subset $Y \subseteq X$ of points, the distance is defined as $d(x,Y) = min_{y \in Y} ||x - y||$. For a subset $Y \subseteq X$ of points, let its \textit{centroid} be given by
\begin{equation*}
\begin{split}
    \text{centroid}(Y) = \frac{1}{|Y|}\sum_{y \in Y} x
\end{split}
\end{equation*}
Let $C =\{c_1,...c_k\}$ be the ser of points and let $Y \subseteq X$. We define the \textit{cost} of $Y$ with respect to $C$ as
\begin{equation*}
\begin{split}
    \phi_Y(C) = \sum_{y \in Y} d^2(y,C) = \sum_{y \in Y} \min_{i=1,...,k}||y-c_i||^2
\end{split}
\end{equation*}


%----------------------------------------------------------------------------------------
%	Algorithm
%----------------------------------------------------------------------------------------

\section*{Algorithm}

%------------------------------------------------

\subsection*{\textbf{$k$-means++($k$) initialization}}
\begin{enumerate}
	\item $C \leftarrow$ sample a point uniformlt at random from $X$ 
	\item \textbf{while} $|C|<k$ \textbf{do}
		\begin{itemize}
		\item Sample $x \in X$ with probability $\frac{d^2(x,C)}{\phi_X(C)}$
		\item $C \leftarrow C \cup \{x\}$ 
		\end{itemize}
	\item \textbf{end while} 
\end{enumerate}

%------------------------------------------------

\subsection*{\textbf{$k$-means||($k,l$) initialization}}
\begin{enumerate}
	\item $C \leftarrow$ sample a point uniformlt at random from $X$ 
	\item $\psi \leftarrow \phi_X(C)$
	\item \textbf{for} $O$(log$\psi$) times \textbf{do}
		\begin{itemize}
		\item $C' \leftarrow$ sample each point $x \in X$ independently with probability $p_x = \frac{l \cdot d^2(x,C)}{\phi_X(C)}$	
		\item $C \leftarrow C \cup C'$ 
		\end{itemize}
	\item \textbf{end for} 
	\item For $x \in C$, set $w_x$ to be the number of points in $X$ closer to $x$ than any point in $C$ 
		\item Recluster the weighted points in $C$ into $k$ clusters
\end{enumerate}

%----------------------------------------------------------------------------------------

\end{document}